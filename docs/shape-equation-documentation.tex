\documentclass{article}

\usepackage{graphicx}
\usepackage{subfigure}
\usepackage{amsmath,amssymb,amsthm}
\usepackage{upgreek}

\newcommand{\JUL}{J$\ddot{\text{u}}$licher }

\begin{document}

\title{Effect of Spontaneous Curvature on a Two-Phase Vesicle}

\date{\today}

\maketitle

\section{Membrane Energy Model and Shape Equations for a Two-Phase Vesicle}
    
    We model a closed lipid vesicle membrane as a two-phase surface
    \[
        \Gamma = \Gamma^{(1)} \cup \Gamma^{(2)}, 
        \qquad 
        \Gamma^{(1)} \cap \Gamma^{(2)} = \partial\Gamma,
    \]
    where $\Gamma^{(1)}$ and $\Gamma^{(2)}$ are smooth surface patches occupied by distinct material phases (phase $1$ and phase $2$), and $\partial\Gamma$ is their common interface (a closed space curve). The membrane energy consists of surface-tension contributions for each phase, bending contributions of Helfrich type for each phase, and a line-tension contribution concentrated at the phase boundary.
    
    \subsection{General Energy Functional}
    
        \paragraph{Surface energy.}
        For each phase $i\in\{1,2\}$, the surface-tension (or stretching) energy is taken to be
        \[
            E_S^{(i)} \;=\; \int_{\Gamma^{(i)}} \tau^{(i)}\, dA,
        \]
        where $\tau^{(i)}$ is the (effective) surface tension on $\Gamma^{(i)}$.
        
        \paragraph{Bending energy (Helfrich form).}
        The bending energy on $\Gamma^{(i)}$ is decomposed into mean-curvature and Gaussian-curvature parts:
        \[
            E_B^{(i)} \;=\; E_M^{(i)} + E_G^{(i)}
            \;=\;
            \int_{\Gamma^{(i)}} \frac{\kappa^{(i)}}{2}\,\bigl(2H - H_0^{(i)}\bigr)^2\, dA
            \;+\;
            \int_{\Gamma^{(i)}} \kappa_G^{(i)}\,K\, dA.
        \]
        Here $H$ and $K$ denote the mean and Gaussian curvatures of the surface, $\kappa^{(i)}$ and $\kappa_G^{(i)}$ are the (phase-dependent) mean and Gaussian bending rigidities, and $H_0^{(i)}$ is the spontaneous curvature of phase $i$. (Throughout, we use the convention that $2H$ is the sum of principal curvatures.)
        
        \paragraph{Line energy.}
        The phase boundary $\partial\Gamma$ carries a line tension $\sigma$, contributing
        \[
            E_L \;=\; \int_{\partial\Gamma} \sigma \, dl.
        \]
        
        \paragraph{Constraints.}
        In typical lipid bilayer regimes, stretching is energetically costly, and the total area of each phase is treated as fixed (or nearly so), while the enclosed volume is constrained by osmotic pressure. We enforce these constraints using Lagrange multipliers:
        \[
            A^{(i)} = \int_{\Gamma^{(i)}} dA \quad \text{(fixed)}, 
            \qquad
            V = \int_{V_\Gamma} dV \quad \text{(fixed)}.
        \]
        Let $\Sigma^{(i)}$ enforce the area constraint on $\Gamma^{(i)}$ and let $P$ enforce the volume constraint. (Physically, $P$ corresponds to the pressure jump across the membrane.) Since $\tau^{(i)}$ and $\Sigma^{(i)}$ enter the Euler--Lagrange equations only through their sum, it is standard to absorb the area multipliers into an effective tension and write
        \[
            \tau^{(i)} \;\leftarrow\; \tau^{(i)} + \Sigma^{(i)}.
        \]
        
        With this convention, the constrained energy functional can be written as
        \begin{equation}
            \label{eq:energy_general}
            E \;=\;
            \sum_{i=1}^2 \int_{\Gamma^{(i)}} 
            \left[
            \tau^{(i)} 
            + \frac{\kappa^{(i)}}{2}\,\bigl(2H - H_0^{(i)}\bigr)^2 
            + \kappa_G^{(i)} K
            \right] dA
            \;+\;
            \int_{\partial\Gamma} \sigma\, dl
            \;+\;
            \int_{V_\Gamma} P\, dV.
        \end{equation}
    
    \subsection{Axisymmetric Geometry}
    
    We restrict attention to axisymmetric vesicle shapes obtained by revolving a planar generating curve about the vertical $z$-axis. Let $s$ denote arclength along the generating curve, with total arclength $s_f$. The generating curve is described by
    \[
        (r(s), z(s)), \qquad s \in [0,s_f],
    \]
    where $r(s)\ge 0$ is the distance from the axis of revolution and $z(s)$ is the height. Let $\psi(s)$ denote the tangent angle measured counterclockwise from the positive $r$-axis, so that the arclength parametrization implies
    \begin{equation}
        \label{eq:kinematics}
        r'(s) = \cos\psi(s),
        \qquad
        z'(s) = \sin\psi(s).
    \end{equation}
    We refer to $s=0$ as the south pole, $s=s_f$ as the north pole, and $s=s_b$ as the phase boundary (the ``neck''), where $0<s_b<s_f$. Phase $1$ occupies $s\in[0,s_b]$ and phase $2$ occupies $s\in[s_b,s_f]$. We denote boundary values at $s=s_b$ by a subscript $b$ (e.g.\ $r_b=r(s_b)$, $\psi_b=\psi(s_b)$).
    
    \paragraph{Area and volume elements.}
    Under revolution, the surface area element and differential volume contribution (for the enclosed volume) are
    \begin{equation}
        \label{eq:area_volume_elements}
        dA = 2\pi r\, ds,
        \qquad
        dV = \pi r^2 \sin\psi \, ds.
    \end{equation}
    
    \paragraph{Curvatures.}
    For an axisymmetric surface of revolution parametrized by arclength, the principal curvatures are
    \[
    k_s = \psi'(s),
    \qquad
    k_\theta = \frac{\sin\psi(s)}{r(s)}.
    \]
    Hence the mean curvature $H$ and Gaussian curvature $K$ are
    \begin{equation}
    \label{eq:HandK}
    H(s) = \frac{1}{2}\left(\psi'(s) + \frac{\sin\psi(s)}{r(s)}\right),
    \qquad
    K(s) = \frac{\sin\psi(s)}{r(s)}\,\psi'(s).
    \end{equation}
    
    \subsection{Axisymmetric Energy}
    
        Using \eqref{eq:area_volume_elements}, the surface-tension and mean-bending terms reduce to one-dimensional integrals over the generating curve on each phase. The Gaussian-curvature terms can be simplified using Gauss--Bonnet: for two phases joined along $\partial\Gamma$, the net Gaussian contribution reduces (up to topological constants that do not affect the shape equations) to a boundary term involving the jump in $\kappa_G$ across the interface. In the axisymmetric setting, this yields an effective interface term proportional to $\cos\psi_b$.
        
        Collecting terms, we write the axisymmetric energy in the form
        \begin{align}
            \label{eq:axisym_energy}
            E
            &=
            \int_{\Gamma^{(1)}} \frac{\kappa^{(1)}}{2}\,\bigl(2H - H_0^{(1)}\bigr)^2\, dA
            \;+\;
            \int_{\Gamma^{(2)}} \frac{\kappa^{(2)}}{2}\,\bigl(2H - H_0^{(2)}\bigr)^2\, dA
            \nonumber\\
            &\quad
            +\;
            2\pi\bigl(\kappa_G^{(2)}-\kappa_G^{(1)}\bigr)\cos\psi_b
            \;+\;
            \tau^{(1)}A^{(1)} + \tau^{(2)}A^{(2)}
            \;+\;
            2\pi\sigma r_b
            \;+\;
            P V.
        \end{align}
        Here $A^{(i)}$ and $V$ are the phase areas and enclosed volume, and $r_b,\psi_b$ are the radius and tangent angle at the interface.
    
    \subsection{Shape Equations as a First-Order System}
    
        Taking the first variation of \eqref{eq:energy_general} (equivalently \eqref{eq:axisym_energy} in the axisymmetric class) yields Euler--Lagrange equations for $(r,z,\psi)$ coupled to curvature variables. A convenient formulation introduces an auxiliary variable $Q(s)$ that serves as a transverse shear force (a first-integral variable conjugate to $H$). On each phase $i\in\{1,2\}$, the equilibrium equations can be written as the first-order system
        \begin{align}
        Q' 
        &=
        -\,Q\,\frac{\cos\psi}{r}
        - \kappa^{(i)}\bigl(2H - H_0^{(i)}\bigr)
        \left[
        2\left(H - \frac{\sin\psi}{r}\right)^2 + H\,H_0^{(i)}
        \right]
        + 2H\,\tau^{(i)} + P,
        \label{eq:Qeq}
        \\
        H' &= \frac{Q}{2\kappa^{(i)}},
        \label{eq:Heq}
        \\
        \psi' &= 2H - \frac{\sin\psi}{r},
        \label{eq:psieq}
        \\
        r' &= \cos\psi,
        \label{eq:req}
        \\
        z' &= \sin\psi,
        \label{eq:zeq}
        \\
        \bigl(\tau^{(i)}\bigr)' &= 0,
        \label{eq:taueq}
        \end{align}
        together with integral-accumulation variables for volume, area, and (optionally) bending-energy bookkeeping:
        \begin{align}
        V' &= \pi r^2 \sin\psi,
        \label{eq:Veq}
        \\
        A' &= 2\pi r,
        \label{eq:Aeq}
        \\
        B' &= \pi r\,\kappa^{(i)}\bigl(2H - H_0^{(i)}\bigr)^2.
        \label{eq:Beq}
        \end{align}
        In \eqref{eq:Qeq}--\eqref{eq:Beq}, the prime denotes differentiation with respect to arclength $s$.
    
    \subsection{Boundary, Continuity, and Interface Matching Conditions}
    
        \paragraph{Pole boundary conditions.}
        Regularity at the poles requires $r=0$ and appropriate smoothness of the axisymmetric surface. A commonly used set of pole conditions is
        \[
        r(0)=0, \qquad \psi(0)=0,
        \qquad
        r(s_f)=0, \qquad \psi(s_f)=\pi,
        \]
        together with force-regularity conditions on $Q$ at the poles (often $Q(0)=Q(s_f)=0$) and a choice of vertical reference (e.g.\ $z(0)=0$). In computations, additional conditions are supplied via the imposed total area fractions and volume constraint through the accumulated variables $A^{(i)}$ and $V$.
        
        \paragraph{Continuity at the phase boundary.}
        The surface itself is continuous across $s=s_b$, so the geometric variables are continuous:
        \[
        r^{(1)}(s_b)=r^{(2)}(s_b)=r_b,
        \qquad
        z^{(1)}(s_b)=z^{(2)}(s_b)=z_b,
        \qquad
        \psi^{(1)}(s_b)=\psi^{(2)}(s_b)=\psi_b.
        \]
        
        \paragraph{Interface matching (force and moment balance).}
        Variation of the energy additionally yields jump conditions at $\partial\Gamma$ that encode balances of forces and bending moments in the presence of line tension and a jump in Gaussian rigidity. In the axisymmetric formulation, these can be written as
        \begin{align}
        \sigma\,\frac{\sin\psi_b}{r_b}
        &=
        2\kappa^{(2)}\bigl(H'_b\bigr)^{(2)}
        -
        2\kappa^{(1)}\bigl(H'_b\bigr)^{(1)},
        \label{eq:match1}
        \\[2mm]
        \bigl(\kappa_G^{(2)}-\kappa_G^{(1)}\bigr)\frac{\sin\psi_b}{r_b}
        &=
        \kappa^{(1)}\bigl(2H_b^{(1)}-H_0^{(1)}\bigr)
        -
        \kappa^{(2)}\bigl(2H_b^{(2)}-H_0^{(2)}\bigr),
        \label{eq:match2}
        \\[2mm]
        -\sigma\,\frac{\cos\psi_b}{r_b}
        &=
        \kappa^{(2)}\bigl(2H_b^{(2)}-H_0^{(2)}\bigr)
        \left(
        H_b^{(2)}-\frac{\sin\psi_b}{r_b}+\frac{H_0^{(2)}}{2}
        \right)
        \nonumber\\
        &\quad
        -
        \kappa^{(1)}\bigl(2H_b^{(1)}-H_0^{(1)}\bigr)
        \left(
        H_b^{(1)}-\frac{\sin\psi_b}{r_b}+\frac{H_0^{(1)}}{2}
        \right)
        -
        \tau^{(2)}+\tau^{(1)}.
        \label{eq:match3}
        \end{align}
        Equations \eqref{eq:match1}--\eqref{eq:match3} represent, respectively, a normal-force balance involving the line tension, a bending-moment jump condition involving the Gaussian rigidity contrast, and a tangential/meridional force balance involving the jump in effective tension.
    
    \subsection{Non-dimensionalization}
    
        Let $R_0$ be the characteristic length defined by the total membrane area:
        \[
        A^{(1)}+A^{(2)} = 4\pi R_0^2.
        \]
        We non-dimensionalize lengths by $R_0$ and scale the total energy by $2\pi\sigma R_0$. Introduce the dimensionless area fractions
        \[
        x^{(i)} := \frac{A^{(i)}}{4\pi R_0^2},
        \qquad
        x^{(1)}+x^{(2)}=1,
        \]
        and the reduced volume
        \[
        v := \frac{V}{\frac{4\pi}{3}R_0^3}.
        \]
        Define dimensionless material and constraint parameters
        \[
        \hat{\kappa}^{(i)} := \frac{\kappa^{(i)}}{\sigma R_0},
        \qquad
        \hat{H}_0^{(i)} := R_0 H_0^{(i)},
        \qquad
        \hat{\tau}^{(i)} := \frac{R_0}{\sigma}\tau^{(i)},
        \qquad
        \hat{P} := \frac{R_0^2}{\sigma}P,
        \qquad
        \hat{\kappa}_G := \frac{\kappa_G^{(2)}-\kappa_G^{(1)}}{\sigma R_0}.
        \]
        For the state variables, we set
        \[
        \hat{s}=\frac{s}{R_0},
        \quad
        \hat{r}=\frac{r}{R_0},
        \quad
        \hat{z}=\frac{z}{R_0},
        \quad
        \hat{H}=R_0 H,
        \quad
        \hat{Q}=\frac{R_0}{\sigma}Q,
        \quad
        \hat{\psi}=\psi,
        \]
        and similarly scale any accumulated energies if desired (e.g.\ $\hat{B} := B/(2\pi\sigma R_0)$).
        
        In these variables, the axisymmetric first-order system \eqref{eq:Qeq}--\eqref{eq:Beq} becomes (dropping hats on $\psi$ for readability and writing $(\cdot)'=d/d\hat{s}$)
        \begin{align}
        \hat{Q}' &=
        - \hat{Q}\,\frac{\cos\psi}{\hat{r}}
        - \hat{\kappa}^{(i)}\bigl(2\hat{H}-\hat{H}_0^{(i)}\bigr)
        \left[
        2\left(\hat{H}-\frac{\sin\psi}{\hat{r}}\right)^2 + \hat{H}\hat{H}_0^{(i)}
        \right]
        + 2\hat{H}\,\hat{\tau}^{(i)} + \hat{P},
        \label{eq:nd_Q}
        \\
        \hat{H}' &= \frac{\hat{Q}}{2\hat{\kappa}^{(i)}},
        \label{eq:nd_H}
        \\
        \psi' &= 2\hat{H} - \frac{\sin\psi}{\hat{r}},
        \label{eq:nd_psi}
        \\
        \hat{r}' &= \cos\psi,
        \qquad
        \hat{z}' = \sin\psi,
        \qquad
        (\hat{\tau}^{(i)})' = 0,
        \label{eq:nd_geom}
        \\
        v' &= \frac{3}{4}\,\hat{r}^2 \sin\psi,
        \qquad
        (x^{(i)})' = \frac{1}{2}\,\hat{r},
        \qquad
        \hat{B}' = \frac{1}{2}\,\hat{r}\,\hat{\kappa}^{(i)}\bigl(2\hat{H}-\hat{H}_0^{(i)}\bigr)^2.
        \label{eq:nd_integrals}
        \end{align}
        The interface conditions \eqref{eq:match1}--\eqref{eq:match3} reduce to the dimensionless form
        \begin{align}
        \frac{\sin\psi_b}{\hat{r}_b} &= \hat{Q}_b^{(2)}-\hat{Q}_b^{(1)},
        \label{eq:nd_match1}
        \\
        \hat{\kappa}_G\,\frac{\sin\psi_b}{\hat{r}_b}
        &=
        \hat{\kappa}^{(1)}\bigl(2\hat{H}_b^{(1)}-\hat{H}_0^{(1)}\bigr)
        -
        \hat{\kappa}^{(2)}\bigl(2\hat{H}_b^{(2)}-\hat{H}_0^{(2)}\bigr),
        \label{eq:nd_match2}
        \\
        -\frac{\cos\psi_b}{\hat{r}_b}
        &=
        \hat{\kappa}^{(2)}\bigl(2\hat{H}_b^{(2)}-\hat{H}_0^{(2)}\bigr)
        \left(
        \hat{H}_b^{(2)}-\frac{\sin\psi_b}{\hat{r}_b}+\frac{\hat{H}_0^{(2)}}{2}
        \right)
        \nonumber\\
        &\quad
        -
        \hat{\kappa}^{(1)}\bigl(2\hat{H}_b^{(1)}-\hat{H}_0^{(1)}\bigr)
        \left(
        \hat{H}_b^{(1)}-\frac{\sin\psi_b}{\hat{r}_b}+\frac{\hat{H}_0^{(1)}}{2}
        \right)
        -\hat{\tau}^{(2)}+\hat{\tau}^{(1)}.
        \label{eq:nd_match3}
        \end{align}
    
    \subsection{Area-Preserving Reparameterization}
    
        For numerical purposes, it is often advantageous to reparameterize the generating curve so that the independent variable directly tracks surface area. Define a new parameter $S\in[0,S_f]$ by
        \begin{equation}
        \label{eq:dsdS}
        \frac{ds}{dS} = \frac{\sin S}{r},
        \qquad
        s(0)=0.
        \end{equation}
        Then the surface area accumulated up to $s_0=s(S_0)$ satisfies
        \[
        A(s_0)=2\pi\int_0^{s_0} r(s)\,ds
        =2\pi\int_0^{S_0} r(S)\,\frac{ds}{dS}\,dS
        =2\pi\int_0^{S_0}\sin S\,dS
        =2\pi\bigl(1-\cos S_0\bigr).
        \]
        In particular, if $s_0=s_f$ corresponds to $S_0=S_f$, then
        \[
        A(s_f)=A^{(1)}+A^{(2)}=4\pi
        \quad\Longrightarrow\quad
        4\pi = 2\pi(1-\cos S_f)
        \quad\Longrightarrow\quad
        S_f=\pi,
        \]
        so the domain becomes $S\in[0,\pi]$, and solutions may be viewed as deformations of a right half-unit semicircle in this area-based coordinate.
        
        \paragraph{Phase fraction and interface location.}
        Let $S^*$ denote the $S$-location of the phase boundary. Using
        \[
        x^{(1)}=\frac{A^{(1)}}{A^{(1)}+A^{(2)}}=\frac{A(S^*)}{A(\pi)}
        =\frac{2\pi(1-\cos S^*)}{4\pi}
        =\frac12(1-\cos S^*),
        \]
        we obtain
        \begin{equation}
        \label{eq:Sstar}
        S^* = \arccos\bigl(1-2x^{(1)}\bigr).
        \end{equation}
        
        \paragraph{Transformed ODE system.}
        Writing $\dot{(\cdot)} = d/dS$, the chain rule gives $\dot{y} = y'\,(ds/dS)$, hence the system
        \eqref{eq:Qeq}--\eqref{eq:Beq} transforms to
        \begin{align}
        \dot{Q} &=
        \left[
        -\,Q\,\frac{\cos\psi}{r}
        - \kappa^{(i)}\bigl(2H - H_0^{(i)}\bigr)
        \left\{
        2\left(H-\frac{\sin\psi}{r}\right)^2 + H\,H_0^{(i)}
        \right\}
        + 2H\,\tau^{(i)} + P
        \right]\frac{\sin S}{r},
        \label{eq:S_Q}
        \\
        \dot{H} &= \frac{Q}{2\kappa^{(i)}}\,\frac{\sin S}{r},
        \qquad
        \dot{\psi} = \left(2H-\frac{\sin\psi}{r}\right)\frac{\sin S}{r},
        \label{eq:S_Hpsi}
        \\
        \dot{r} &= \cos\psi\,\frac{\sin S}{r},
        \qquad
        \dot{z} = \sin\psi\,\frac{\sin S}{r},
        \qquad
        \dot{\tau}^{(i)}=0,
        \label{eq:S_geom}
        \\
        \dot{s} &= \frac{\sin S}{r},
        \qquad
        \dot{V} = \pi r^2\sin\psi\,\frac{\sin S}{r},
        \qquad
        \dot{A} = 2\pi r\,\frac{\sin S}{r}=2\pi\sin S,
        \label{eq:S_integrals1}
        \\
        \dot{B} &= \pi r\,\kappa^{(i)}\bigl(2H-H_0^{(i)}\bigr)^2\,\frac{\sin S}{r}
        =\pi\kappa^{(i)}\bigl(2H-H_0^{(i)}\bigr)^2\sin S.
        \label{eq:S_integrals2}
        \end{align}
        (If working in the dimensionless variables, replace $(r,H,Q,\kappa^{(i)},\tau^{(i)},P)$ by their hatted counterparts and interpret $S$ on $[0,\pi]$ as above.)
    
    \subsection{Parameterization and Phase Mapping}
    
        We model equilibrium shapes of two-phase vesicles by solving a three-point boundary-value problem (BVP) written in arclength $s$:
        \[
        s \in [0,s^*]\ \cup\ [s^*,s_F],
        \]
        where $s_F$ is the total arclength and $s^*$ is the arclength location of the phase interface (neck). The $\alpha$-phase occupies $[0,s^*]$ and the $\beta$-phase occupies $[s^*,s_F]$.
        
        To preserve surface area in the independent variable, we introduce the area-based change of variables
        \[
        \frac{ds}{dS}=\frac{\sin S}{r(S)},
        \]
        so that the problem is posed on
        \[
        S \in [0,S^*]\ \cup\ [S^*,\pi],
        \]
        where $S=0$ corresponds to the $\alpha$-south pole, $S=\pi$ corresponds to the $\beta$-north pole, and $S=S^*$ corresponds to the phase interface.
        
        For numerical convenience, we solve both phases simultaneously as a two-point BVP on a common parameter $t\in[0,\pi]$ by introducing affine maps
        \[
        S_\alpha(t)=\frac{S^*}{\pi}\,t,
        \qquad
        S_\beta(t)=\pi-\frac{\pi-S^*}{\pi}\,t.
        \]
        Thus:
        \[
        \alpha\text{-south pole: } S_\alpha(0)=0,\qquad
        \beta\text{-north pole: } S_\beta(0)=\pi,
        \]
        \[
        \alpha\text{-neck: } S_\alpha(\pi)=S^*,\qquad
        \beta\text{-neck: } S_\beta(\pi)=S^*.
        \]
        
        In the implementation, we evolve separate state vectors on the two mapped domains,
        \[
        Y_\alpha = \begin{bmatrix} Q_\alpha\\ H_\alpha\\ \psi_\alpha\\ r_\alpha\\ z_\alpha\\ L_\alpha\\ s_\alpha\\ V_\alpha\\ E_\alpha \end{bmatrix},
        \qquad
        Y_\beta = \begin{bmatrix} Q_\beta\\ H_\beta\\ \psi_\beta\\ r_\beta\\ z_\beta\\ L_\beta\\ s_\beta\\ V_\beta\\ E_\beta \end{bmatrix},
        \qquad
        Y=\begin{bmatrix} Y_\alpha\\ Y_\beta \end{bmatrix}.
        \]
        Here $L=\tau$ is the (piecewise constant) effective tension on each phase, and $E$ is an accumulated bending-energy quantity used for bookkeeping.
        
        
        \medskip
        This completes the model specification: the equilibrium vesicle shapes are obtained by solving the two-phase boundary-value problem consisting of the axisymmetric shape system on $[0,s_b]$ and $[s_b,s_f]$ (or on $[0,S^*]$ and $[S^*,\pi]$ after reparameterization), together with pole regularity, geometric continuity, and interface matching conditions \eqref{eq:match1}--\eqref{eq:match3}.

    
    \subsection{Pole Taylor Expansions for Singular Ratios}
        
        In the area-based parameter $S\in[0,\pi]$, the axisymmetric shape equations contain ratios
        \[
            \frac{\sin\psi}{r},\qquad \frac{\cos\psi}{r},\qquad \frac{\sin S}{r},
        \]
        which are singular at the poles because $r\to 0$ as $S\to 0$ (south pole) and $S\to\pi$ (north pole).
        To evaluate the ODE system robustly near the poles, we replace these ratios by Taylor truncations
        in a local pole-distance variable.
        
        \paragraph{Local pole-distance variable.}
        Define
        \[
            \varepsilon =
            \begin{cases}
            S, & S\approx 0\quad\text{(south pole)},\\[2pt]
            \pi-S, & S\approx \pi\quad\text{(north pole)},
            \end{cases}
            \qquad \varepsilon\downarrow 0.
        \]
        Let
        \[
            H_p := H(\text{pole}),\qquad
            H''_p := \left.\dfrac{d^2H}{dS^2}\right|_{\text{pole}}.
        \]
        Smooth axisymmetry implies $r(\varepsilon)=\varepsilon+O(\varepsilon^3)$ and $\psi$ approaches its pole value
        linearly, so the following expansions hold.
        
        \paragraph{Regularity identities for $Q'(0)$ and $H''_p$.}
        In nondimensional variables (with phase-dependent $k=\hat\kappa^{(i)}$ and $H_0=\hat H_0^{(i)}$),
        regularity of the $Q$-equation at a pole yields a finite pole value for $dQ/dS$:
        \begin{equation}
            \label{eq:Qp_pole}
            Q'(0)
            =
            H_p\,L+\frac{P}{2}-k\,H_0\,H_p^2+\frac{k}{2}\,H_p\,H_0^2,
        \end{equation}
        where $L=\hat\tau^{(i)}$ is the (constant) effective tension on the corresponding phase and $P=\hat P$.
        Since 
        \[
            \frac{dH}{dS} = \frac{Q}{2k}\frac{\sin S}{r}
            \quad\text{and}\quad 
            \frac{\sin S}{r}\to 1\quad \text{at the pole,}
        \] 
        we obtain
        \begin{equation}
            \label{eq:Hpp_pole}
            H''_p = \frac{Q'(0)}{2k}.
        \end{equation}
        
        \paragraph{Taylor expansions.}
        Using \eqref{eq:Hpp_pole}, the ratios admit the pole expansions
        \begin{align}
            \frac{\sin\psi}{r}
            &=
            H_p + \frac{H''_p}{4}\,\varepsilon^2 + O(\varepsilon^4),
            \label{eq:sinpsi_over_r_exp}
            \\
            \frac{\sin S}{r}
            &=
            1-\frac{1-H_p^2}{8}\,\varepsilon^2 + O(\varepsilon^4).
            \label{eq:sinS_over_r_exp}
        \end{align}
        These have no odd-power corrections; hence the first-, second-, and third-degree truncations in $\varepsilon$
        agree at degrees $1$ and $2$ as:
        \[
            \text{(deg 1)}\quad
            \frac{\sin\psi}{r}\approx H_p,\quad \frac{\sin S}{r}\approx 1,
        \]
        \[
            \text{(deg 2/3)}\quad
            \frac{\sin\psi}{r}\approx H_p+\frac{H''_p}{4}\varepsilon^2,\quad
            \frac{\sin S}{r}\approx 1-\frac{1-H_p^2}{8}\varepsilon^2.
        \]
        
        \paragraph{Pole-safe evaluation of the singular product $Q\cos\psi/r$.}
        Although $\cos\psi/r\sim \pm 1/\varepsilon$ is singular, the product $Q\cos\psi/r$ has a finite pole limit because
        $Q(\varepsilon)=Q'(0)\varepsilon+O(\varepsilon^3)$. In particular,
        \begin{equation}
            \label{eq:Qcos_over_r_limit}
            \frac{Q\cos\psi}{r} = Q'(0) + O(\varepsilon^2),
        \end{equation}
        with $Q'(0)$ given by \eqref{eq:Qp_pole}. In computations, we therefore replace the singular term
        $-Q\cos\psi/r$ in the $Q$-equation by the pole-safe approximation $-Q'(0)$ (optionally with $O(\varepsilon^2)$ corrections).

\end{document}